In this and the next section, we investigate by computation how general this
phenomenon is.  Following up on the experiment of \cite{leveque2003},
one may ask whether solitary waves are observed if some of the parameters are
changed.  In this section we investigate whether similar results are obtained
for other types of periodic media.

\subsection{Smoothly varying media}
We now consider the case of a smoothly varying medium, with
\begin{subequations}
\label{eq:sinmat}
\begin{align}
\rho(x) = & a + b \sin\left(2\pi \frac{x}{\delta}\right) \\
K(x) = & a + b \sin\left(2\pi\left(\frac{x}{\delta}+\theta\right)\right)
\end{align}
\end{subequations}
and exponential stress-strain relation \ref{expstress}.
We first consider the case $\theta=0$, so that $K(x)=\rho(x)$ and thus
the linearized sound speed is constant.  Taking $a=5/2$, $b=3/2$, we find solitary
wave behavior similar to the piecewise constant case, as shown in Figure
\ref{fig:sinoton}.  It can be shown that the coefficients of the dissipative
terms in the homogenized equations vanish in this case.

Numerical tests of time-reversibility and entropy evolution also seem
to indicate that shocks do not form, just as in the piecewise-constant
medium case.  If the amplitude of the impedance variation is reduced
sufficiently, shocks are observed to form and the solitary waves are
replaced by incoherent fluctuations.
\begin{figure}
\begin{center}
\subfigure{\includegraphics[width=3.5in]{figures/sinoton_zoomout.eps}}
\subfigure{\includegraphics[width=3.5in]{figures/sinoton_closeup.eps}}
\subfigure{\includegraphics[width=3.5in]{figures/sinoton_params.eps}}
\caption{Results for the medium \eqref{eq:sinmat} with $\theta=0$. \label{fig:sinoton}}
\end{center}
\end{figure}

Next we take $\theta=1/2$, so that the fluctuations in the density and the
bulk modulus are precisely out of phase (and thus the linearized sound speed 
varies dramatically).  Again, it can be shown that the coefficients of the 
dissipative terms in the homogenized equations vanish in this case.  However, the 
observed behavior (shown in Figure \ref{fig:oop}) is quite different from 
the previous example.  Now the
solution evolves in a manner similar to Burgers equation, with the nonlinearity
dominating.  The dispersion, rather than leading to solitary waves, appears only
to generate noise-like oscillations.

\begin{figure}
\begin{center}
\subfigure{\includegraphics[width=3.5in]{figures/oop.eps}}
\subfigure{\includegraphics[width=3.5in]{figures/oop_stress.eps}}
\subfigure{\includegraphics[width=3.5in]{figures/oop_params.eps}}
\caption{Strain (left) and stress (right) for the medium \eqref{eq:sinmat} with $\theta=1/2$. \label{fig:oop}}
\end{center}
\end{figure}

%Now we consider the case $\theta=1/4$.  In this case, the coefficients
%of the dissipative homogenized terms are nonzero.  Solitary waves are
%observed, but there is a lot of 'junk' between and behind them, and the leading
%one has two peaks.
%
%\begin{figure}
%\begin{center}
%\subfigure{\includegraphics[width=3.5in]{figures/hop.eps}}
%\subfigure{\includegraphics[width=3.5in]{figures/hop_stress.eps}}
%\subfigure{\includegraphics[width=3.5in]{figures/hop_params.eps}}
%\caption{Strain (left) and stress (right) for the medium \eqref{eq:sinmat} with $\theta=1/4$. \label{fig:hop}}
%\end{center}
%\end{figure}
%
%Finally, consider the case $\theta=0$ but with the period of the density
%variation doubled (while the bulk modulus variation remains the same).
%In this case, dispersion seems to dominate and little coherent structure
%remains after even a short time.
%
%\begin{figure}
%\begin{center}
%\subfigure{\includegraphics[width=3.5in]{figures/dp.eps}}
%\subfigure{\includegraphics[width=3.5in]{figures/dp_stress.eps}}
%\subfigure{\includegraphics[width=3.5in]{figures/dp_params.eps}}
%\caption{Strain (left) and stress (right) for the medium \eqref{eq:sinmat} with $\theta=0$ and the period of the density variation equal to twice the period of the bulk modulus variation. \label{fig:dp}}
%\end{center}
%\end{figure}



\subsection{Randomly perturbed media}
In this section we consider the effect of random defects in the media.
Our goal is to assess how sensitive regularization and solitary wave 
formation are to the precise periodic nature of the medium.  This may
be important in efforts to experimentally realize these waves.

\subsubsection{Perturbed interface locations}
  We consider an LY medium but with $\alpha$ in layer $i$ given by
  \[\tilde{\alpha}_i = \frac{1}{2}(1+\mu \gamma_i)\]
  where the $\gamma_i$ are taken from a random Gaussian distribution
  with unit standard deviation.  We consider two cases: $\mu=0.01$
  and $\mu=0.1$.  Results are shown in figure \Fig{randint}.

  \begin{figure}  \subfigure[Perturbed by 1\%]{\includegraphics[width=3in]{figures/random_steg_1percent.png}}
  \subfigure[Perturbed by 10\%]{\includegraphics[width=3in]{figures/random_steg_10percent.png}}
  \caption{Effect of perturbing interface locations.\label{fig:randint}}
  \end{figure}

\subsubsection{Perturbed material parameters}
  We consider an LY medium but with $K_B$ in layer $i$ given by
  \[\tilde{K}_{B,i} = 4(1+\mu \gamma_i)\]
  where the $\gamma_i$ are taken from a random Gaussian distribution
  with unit standard deviation.  We consider two cases: $\mu=0.01$
  and $\mu=0.25$.  Results are shown in figure \Fig{randK}.  The
  coherence of the solitary waves appears to be less sensitive to
  this type of perturbations.

  \begin{figure}  \subfigure[Perturbed by 1\%]{\includegraphics[width=3in]{figures/random_steg_val_1percent.png}}
  \subfigure[Perturbed by 25\%]{\includegraphics[width=3in]{figures/random_steg_val_25percent.png}}
  \caption{Effect of perturbing bulk modulus values uniformly in each layer.\label{fig:randK}}
  \end{figure}


