One interesting feature of the stegotons is that they are always observed
to travel faster than the homogenized sound speed of the medium.
I have been able to explain this property by analysis of a simplified 
system of homogenized equations.  

By considering a layered medium with two alternating materials and
introducing as a small parameter $\delta$ the layer thickness, one can derive
effective homogenized equations, to any order in $\delta$.  The 
homogenized equations are written in terms of $\sigma$ and $u$, since
these variables are continuous across material interfaces (whereas 
$\epsilon,\rho u$ are not).  By computational
experiment, I have found that the qualitative behavior of the system
can be reproduced by considering the zeroth-order equations plus one of
the next order terms.  For a simple exponential stress-strain relation,
these are
\begin{subequations}
\label{homsimp}
\begin{align}
\sigma_t-(\sigma+1)\hat{K}u_x & = -(\sigma+1)\hat{K}Cu_{xxx} \\
\hat{\rho} u_t - \sigma_x & = 0.
\end{align}
\end{subequations}
Here the quantities with hats are effective homogenized parameters
and $C$ is the parameter $C_{12}$ from \cite{leveque2003}.
The terms on the left correspond to the original equations with 
homogenized coefficients.  Even this system with just one additional
term supports solitary waves.  To see this, we use a traveling wave
ansatz: 
\begin{subequations}
\begin{align}
u(x,t) = & U(x-Vt) \\
(\sigma(x,t)+1) = & W(x-Vt).
\end{align}
\end{subequations}
Here $V$ is a constant velocity.  Substituting these into (\ref{homsimp})
gives (dropping the hats for simplicity)
\begin{subequations}
\begin{align}
-VW' & = K(WU'-CWU''') \\
-\rho V U' & = W'.
\end{align}
\end{subequations}
Using the second equation we can eliminate $U$ in the first to obtain
\begin{subequations}
\begin{align}
W' & = \frac{K}{\rho V^2}(WW'-CWW''')
\end{align}
\end{subequations}
Since
\begin{subequations}
\begin{align}
WW' & = \left(\frac{1}{2}W^2\right)' \\
WW''' & = (WW'')' - W'W'' = \left(WW''-\frac{1}{2}W'^2\right)',
\end{align}
\end{subequations}
we can integrate by parts to obtain (defining $\alpha = \frac{K}{\rho V^2}$)
$$
W = \alpha \left( \frac{1}{2}W^2 - C\left(WW''-\frac{1}{2}W'^2\right) \right) + \gamma
$$
The constant of integration $\gamma$ is determined by the physical boundary
conditions at $\infty$: $W\rightarrow 1, W'\rightarrow 0$.  This yields
$$
\gamma = 1-\frac{\alpha}{2}.
$$
Rearranging (and making the physically justified assumption $W\ne0$), we have
$$
W'' = \frac{W'^2}{2W} + \frac{W}{2C} - \frac{1}{\alpha C} \left(1 - \frac{2- \alpha}{2W}\right).
$$
This can be rewritten as a first order system by setting $w_1=W,w_2=W'$:
\begin{subequations}
\label{odes}
\begin{align}
w_1' & = w_2 \\
w_2' & = \frac{w_2^2}{2w_1} + \frac{w_1}{2C} - \frac{1}{\alpha C} \left(1 -\frac{2-\alpha}{2w_1}\right).
\end{align}
\end{subequations}
Setting $w_1'=w_2'=0$, we find that the equilibria of this system occur
at $(1,0)$ and $(\frac{2-\alpha}{\alpha},0)$.
The Jacobian of system (\ref{odes}) is
\be
J = \left( \begin{array}{cc}
0 & 1 \\ \frac{w_2^2}{2w_1^2} + \frac{1}{2C} - \frac{2-\alpha}{2\alpha Cw_1^2} & \frac{w_2}{w_1}
\end{array} \right)
\ee


The resulting system has two equilibria, the nature of which
depends on the value of a single parameter, $M$.  This parameter turns out
to be the mach number of the wave (i.e., its velocity divided by the
homogenized sound speed).  If $M>1$, then
one equilibrium point, which corresponds to physical boundary conditions
at infinity, is a saddle, while the other is a center.  The system thus
has a homoclinic connection, giving rise to solitary waves.  If $M<1$,
the saddle point becomes a center, and vice versa.  In this case,
the saddle point corresponds to unphysical boundary conditions (the
corresponding solitary wave would have an infinite amount of energy),
and no stegotons are observed (contrary to intuition based on their name,
stegotons are supersonic!).  The nature of the phase plane for the
two cases is shown in Figure \ref{fig:pptopo}.

\begin{figure}
\begin{pspicture}(0,0)(4,5)
  \rput[c](2.5,4.5){$M>1$}
  \psline[arrows=<->](0,2)(4,2)
  \rput[c](4.25,2){{\color{black}{\small{$W$}}}}
  \psline[arrows=<->](0.5,0)(0.5,4)
  \rput[c](0.5,4.25){{\color{black}{\small{$W'$}}}}
  \pscircle[fillstyle=solid,fillcolor=black,linestyle=dashed](2.75,2){0.1}
  \pscircle[fillstyle=solid,fillcolor=black,linestyle=dashed](1.0,2){0.1}
  \psline[linecolor=red,arrows=->,linearc=0.8,linewidth=.05](1.0,2.0)(2.5,3.0)(3.15,2.75)(3.5,2.)(3.15,1.25)(2.5,1.0)(1.0,2.0)
  \rput[c](1.0,1.5){(1,0)}
\end{pspicture}
\hspace{1cm}
\begin{pspicture}(0,0)(4,5)
  \rput[c](2.5,4.5){$M<1$}
  \psline[arrows=<->](0,2)(4,2)
  \rput[c](4.25,2){{\color{black}{\small{$W$}}}}
  \psline[arrows=<->](0.5,0)(0.5,4)
  \rput[c](0.5,4.25){{\color{black}{\small{$W'$}}}}
  \pscircle[fillstyle=solid,fillcolor=black,linestyle=dashed](1.0,2){0.1}
  \pscircle[fillstyle=solid,fillcolor=black,linestyle=dashed](3.0,2){0.1}
  \psline[linecolor=red,arrows=->,linearc=0.62,linewidth=.05](3.0,2.0)(1.5,3.0)(1.15,2.75)(0.65,2.)(1.15,1.25)(1.5,1.0)(3.0,2.0)
\end{pspicture}
\caption{Phase plane topology for stationary waves. \label{fig:pptopo}}
\end{figure}

By integrating higher-order versions of the homogenized equations
(starting from a point very near the saddle), it is possible to 
reproduce the shape and the amplitude-velocity relation of the stegotons as
well.  My results from this agree very well with those of \cite{leveque2003}.
A more interesting wave results from using a starting point well
inside the homoclinic connection.  Since trajectories interior to
the homoclinic connection are periodic (see Figure \ref{fig:ppss}), 
this leads to periodic waves that should travel through the layered 
medium without changing shape.  We will refer to these waves as stegosoids.  
Numerical simulations using WENOCLAW show that this is indeed the case.
The first three subfigures of Figure \ref{fig:stegosoid} show numerically
integrated waveforms, while the fourth subfigure shows results of using 
such an integrated waveform as a boundary condition for a time-dependent
simulation.
\begin{figure}
\centering
\begin{pspicture}(0,0)(4,5)
  \psline[arrows=<->](0,2)(4,2)
  \rput[c](4.25,2){{\color{black}{\small{$W$}}}}
  \psline[arrows=<->](0.5,0)(0.5,4)
  \rput[c](0.5,4.25){{\color{black}{\small{$W'$}}}}
  \psline[linecolor=red,arrows=->,linearc=0.8,linewidth=.05](1.0,2.0)(2.5,3.0)(3.15,2.75)(3.5,2.)(3.15,1.25)(2.5,1.0)(1.0,2.0)
  \pscircle[fillstyle=solid,fillcolor=black,linestyle=dashed](2.5,2){0.1}
  \pscircle[fillstyle=solid,fillcolor=black,linestyle=dashed](1.0,2){0.1}
  \pscircle[linecolor=black,linestyle=dashed](2.5,2){0.3}
  \pscircle[linecolor=black,linestyle=dashed](2.5,2){0.5}
  \pscircle[linecolor=black,linestyle=dashed](2.5,2){0.7}
\end{pspicture}
\caption{Stegosoids in the phase plane. \label{fig:ppss}}
\end{figure}

\begin{figure}
\subfigure{\includegraphics[width=2.5in]{figures/stegosoid1.eps}}
\subfigure{\includegraphics[width=2.5in]{figures/stegosoid2.eps}}
\subfigure{\includegraphics[width=2.5in]{figures/stegosoid3.eps}}
\subfigure{\includegraphics[width=2.5in]{figures/stegosoidfv.eps}}
\caption{Stegosoids of various wavelengths. \label{fig:stegosoid}}
\end{figure}

I have also begun to answer the important question of whether stegotons
are stable in higher dimensions.
