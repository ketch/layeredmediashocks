In order to study the stegotons in terms of effective medium theory,
we can derive a system 
of effective or homogenized equations that describe the
evolution of (spatially) slowly varying waves \cite{leveque2003}.  
This is done by first changing
variables to write \eqref{nel_pde} as a pair of evolution
equations for $\sigma$ and $u$, which are continuous even
across material interfaces:
\begin{subequations}
\begin{align} \label{nel_inv}
\sigma_t - K(x)G(\sigma)u_x & = 0 \\
\rho(x) u_t - \sigma_x & = 0.
\end{align}
\end{subequations}
Here $G(\sigma)=\sigma_\epsilon(\epsilon,x)/K(x)$.  In the 
case of the exponential stress-strain relationship 
\eqref{expstress}, $G(\sigma)=\sigma+1$.

By assuming that the solution quantities $u, \sigma$ vary slowly
on the scale of the variation in the medium, LeVeque \& Yong derived
effective homogenized equations for this system.    The essential idea is to suppose
that the variation of $u$ and $\sigma$ within a single period is
much smaller than the large-scale variation.  Taking the ratio
of these two as a small parameter $\delta$, the following
homogenized effective equations are obtained:
\begin{align}
u_t & = \frac{1}{\rhomean}\left(\sigma_x-\delta C_{21}\sigma_{xx} 
              - \delta^2 C_{22}\sigma_{xxx} \cdots \right) \\
\sigma_t & = \Kmean\left(G(\sigma)u_x - \delta C_{11}G(\sigma)u_{xx} 
              - \delta^2 \left(C_{12}G(\sigma)u_{xxx} 
                    + C_{13} G'(\sigma)\sigma_x u_{xx} 
                    + C_{13} \frac{1}{2} G''(\sigma)\sigma_x^2 u_x\right)
                    + \cdots \right)
\end{align}
Here the quantities in brackets are spatial averages:
\begin{align}
\rhomean & = \alpha \rho_A + (1-\alpha) \rho_B &
\Kmean   & = \left( \frac{\alpha}{K_A}+\frac{1-\alpha}{K_B} \right)^{-1}
\end{align}
and the constants $C_{ij}$ are functionals of the material parameter
functions $\rho(x), K(x)$.  
For the case of a piecewise-constant bilayered medium,
the coefficients in \eqref{eq:stego-homog} are given by
(note that the equation for $C_{13}$ contains a small typo in
\cite{leveque2003}):
\begin{subequations} \label{eq:stego-coeffs}
\begin{align}
C_{11} & = C_{21}=0 \\
C_{12} & = -\frac{1}{12}\alpha^2(1-\alpha)^2 
            \frac{(\rho_A-\rho_B)(Z_A^2-Z_B^2)}{K_A K_B\Kinvmean \rhomean^2} \\
C_{22} & = -\frac{1}{12}\alpha^2(1-\alpha)^2 
            \frac{(K_A-K_B)(Z_A^2-Z_B^2)}{K_A^2 K_B^2(\Kinvmean)^2 \rhomean} \\
C_{13} & = -\frac{1}{12}\alpha^2(1-\alpha)^2 
            \frac{\rhomean^2(K_A-K_B)^2+(Z_A^2-Z_B^2)^2}
                 {K_A^2 K_B^2 (\Kinvmean)^2 \rhomean^2}.
\end{align}
\end{subequations}
For any piecewise constant medium (as well
as some other media we will consider), it turns out that the coefficients
$C_{11}, C_{21}$, and all other coefficients of even-derivative terms,
vanish.

In the rest of this section, we consider a reduced system obtained
by setting also $C_{22}=C_{13}=0$, and using the exponential
stress relation \eqref{expstress}:
\begin{subequations} \label{homsimp}
\begin{align}
\rhomean u_t - \sigma_x & = 0 \\
\sigma_t - \Kmean(\sigma+1)u_x & = - C\Kmean (\sigma+1)u_{xxx}.
\end{align}
\end{subequations}
Here $C=\delta^2 C_{12}$.
Observe that the terms on the left correspond to the original equations with 
homogenized coefficients.  Even this system, which involves just one additional
term, supports solitary waves.  To see this, we assume a traveling wave 
solution with velocity $\lambda$:
\begin{align*}
u(x,t) = & U(x-\lambda t) & \sigma(x,t)+1 = & W(x-\lambda t).
\end{align*}
Substituting the above into (\ref{homsimp}) gives
\begin{subequations}
\begin{align*}
-\lambda W' & = \Kmean (WU'-CWU''') \\
-\rhomean \lambda  U' & = W'.
\end{align*}
\end{subequations}
Using the second equation we can eliminate $U$ in the first to obtain
\begin{align*}
W' & = \frac{\Kmean }{\rhomean \lambda^2}(WW'-CWW''')
\end{align*}
%Since
%\begin{subequations}
%\begin{align}
%WW' & = \left(\frac{1}{2}W^2\right)' \\
%WW''' & = (WW'')' - W'W'' = \left(WW''-\frac{1}{2}W'^2\right)',
%\end{align}
%\end{subequations}
Integration yields 
$$
W = M^2 \left( \frac{1}{2}W^2 - C\left(WW''-\frac{1}{2}W'^2\right) \right) + \gamma
$$
Here we have defined $M = \frac{\lambda}{\cmean}$, the Mach number of the 
traveling wave relative to the homogenized sound speed for small-amplitude
waves.
The constant of integration $\gamma$ is determined by the physical boundary
conditions at $\infty$: $W\rightarrow 1, W'\rightarrow 0$.  This yields
$ \gamma = 1-\frac{1}{2M^2}$.
Rearranging (and making the physically justified assumption $W\ne0$), we have
$$
W'' = \frac{W'^2}{2W} + \frac{W}{2C} + \frac{1}{C} \left(\frac{2M^2- 1}{2W}-1\right).
$$
This can be rewritten as a first order system by setting $w_1=W,w_2=W'$:
\begin{subequations}
\label{odes}
\begin{align}
w_1' & = w_2 \\
w_2' & = \frac{w_2^2}{2w_1} + \frac{w_1}{2C} - \frac{M^2}{C} \left(\frac{2M^2- 1}{2w_1}-1\right).
\end{align}
\end{subequations}
Rewriting this as a first-order system, we find that it has equilibria at
at $(W,W')=(1,0)$ and $(2M^2-1,0)$.
%The Jacobian of system (\ref{odes}) is
%\begin{align}
%J & = \left( \begin{array}{cc}
%0 & 1 \\ - \frac{w_2^2}{2w_1^2} + \frac{1}{2C} - \frac{2M^2-1}{2 Cw_1^2} & \frac{w_2}{w_1}
%\end{array} \right)
%\end{align}
The nature of these equilibrium points
depends on the value of $M$.  If $M>1$, then
one equilibrium point, which corresponds to the physical boundary conditions
at infinity, is a saddle, while the other is a center.  The system thus
has a homoclinic connection, giving rise to solitary waves.  If $M<1$,
the saddle point becomes a center, and vice versa.  In this case,
the saddle point corresponds to unphysical boundary conditions (the
corresponding solitary wave would have an infinite amount of energy),
and no stegotons are observed.  The nature of the phase plane for the
two cases is shown in Figure \ref{fig:pptopo}.

\begin{figure}
\begin{pspicture}(0,0)(4,5)
  \rput[c](2.5,4.5){$M>1$}
  \psline[arrows=<->](0,2)(4,2)
  \rput[c](4.25,2){{\color{black}{\small{$W$}}}}
  \psline[arrows=<->](0.5,0)(0.5,4)
  \rput[c](0.5,4.25){{\color{black}{\small{$W'$}}}}
  \pscircle[fillstyle=solid,fillcolor=black,linestyle=dashed](2.75,2){0.1}
  \pscircle[fillstyle=solid,fillcolor=black,linestyle=dashed](1.0,2){0.1}
  \psline[linecolor=red,arrows=->,linearc=0.8,linewidth=.05](1.0,2.0)(2.5,3.0)(3.15,2.75)(3.5,2.)(3.15,1.25)(2.5,1.0)(1.0,2.0)
  \rput[c](1.0,1.5){(1,0)}
  \rput[c](2.7,1.5){($2M^2-1$,0)}
\end{pspicture}
\hspace{2cm}
\begin{pspicture}(0,0)(4,5)
  \rput[c](2.5,4.5){$M<1$}
  \psline[arrows=<->](0,2)(4,2)
  \rput[c](4.25,2){{\color{black}{\small{$W$}}}}
  \psline[arrows=<->](0.5,0)(0.5,4)
  \rput[c](0.5,4.25){{\color{black}{\small{$W'$}}}}
  \pscircle[fillstyle=solid,fillcolor=black,linestyle=dashed](2.75,2){0.1}
  \pscircle[fillstyle=solid,fillcolor=black,linestyle=dashed](1.0,2){0.1}
  \psline[linecolor=red,arrows=->,linearc=0.8,linewidth=.05](1.0,2.0)(2.5,3.0)(3.15,2.75)(3.5,2.)(3.15,1.25)(2.5,1.0)(1.0,2.0)
  \rput[c](2.7,1.5){(1,0)}
  \rput[c](1.0,1.5){($2M^2-1$,0)}
\end{pspicture}
\caption{Phase plane topology for stationary waves. \label{fig:pptopo}}
\end{figure}

Although solutions of this model homogenized system differ in details
from solutions of the first-order variable-coefficient system, the
simple model system reproduces key qualitative elements of the full system:
\begin{itemize}
  \item Existence of solitary wave solutions
  \item Solitary waves travel faster than the homogenized linear sound speed
\end{itemize}
%\subsection{Periodic solutions}
%By integrating higher-order versions of the homogenized equations
%(starting from a point very near the saddle), it is possible to 
%reproduce the shape and the amplitude-velocity relation of the stegotons as
%well.  My results from this agree very well with those of \cite{leveque2003}.
%A more interesting wave results from using a starting point well
%inside the homoclinic connection.  Since trajectories interior to
%the homoclinic connection are periodic (see Figure \ref{fig:ppss}), 
%this leads to periodic waves that should travel through the layered 
%medium without changing shape.  We will refer to these waves as stegosoids.  
%Numerical simulations using WENOCLAW show that this is indeed the case.
%The first three subfigures of Figure \ref{fig:stegosoid} show numerically
%integrated waveforms, while the fourth subfigure shows results of using 
%such an integrated waveform as a boundary condition for a time-dependent
%simulation.
%\begin{figure}
%\centering
%\begin{pspicture}(0,0)(4,5)
%  \psline[arrows=<->](0,2)(4,2)
%  \rput[c](4.25,2){{\color{black}{\small{$W$}}}}
%  \psline[arrows=<->](0.5,0)(0.5,4)
%  \rput[c](0.5,4.25){{\color{black}{\small{$W'$}}}}
%  \psline[linecolor=red,arrows=->,linearc=0.8,linewidth=.05](1.0,2.0)(2.5,3.0)(3.15,2.75)(3.5,2.)(3.15,1.25)(2.5,1.0)(1.0,2.0)
%  \pscircle[fillstyle=solid,fillcolor=black,linestyle=dashed](2.5,2){0.1}
%  \pscircle[fillstyle=solid,fillcolor=black,linestyle=dashed](1.0,2){0.1}
%  \pscircle[linecolor=black,linestyle=dashed](2.5,2){0.3}
%  \pscircle[linecolor=black,linestyle=dashed](2.5,2){0.5}
%  \pscircle[linecolor=black,linestyle=dashed](2.5,2){0.7}
%\end{pspicture}
%\caption{Stegosoids in the phase plane. \label{fig:ppss}}
%\end{figure}
%
%\begin{figure}
%\subfigure{\includegraphics[width=2.5in]{figures/stegosoid1.eps}}
%\subfigure{\includegraphics[width=2.5in]{figures/stegosoid2.eps}}
%\subfigure{\includegraphics[width=2.5in]{figures/stegosoid3.eps}}
%\subfigure{\includegraphics[width=2.5in]{figures/stegosoidfv.eps}}
%\caption{Stegosoids of various wavelengths. \label{fig:stegosoid}}
%\end{figure}
