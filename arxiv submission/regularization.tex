It is clear, at least in the case of a piecewise homogeneous medium, 
that at least some kinds of smooth initial data will lead to shocks,
regardless of the material parameters.  For instance,
if the initial data includes sufficiently large gradients, certain
characteristics will intersect before they reach a material interface.
In fact, for any fixed medium it is possible to construct initial data of
arbitrarily small amplitude for which a shock forms in the solution.  To
construct such data, choose a point $x_0$ so that $\sigma(\epsilon,x)$ is
continuous in $x$ in an open neighborhood about $x_0$ and take data
\begin{subequations} \label{ic1}
\begin{align}
\epsilon(x,-\tau) &= \begin{cases} \epsilon_0 &\text{for}~ x<x_0\\
0 &\text{for}~ x\geq x_0, \end{cases}\\
\noalign{\vskip 3pt}
u(x,-\tau) &= \begin{cases} -\epsilon_0\sqrt{\frac{\sigma'(\epsilon_0)}{\rho(x_0)}}
 &\text{for}~ x<x_0\\
0 &\text{for}~ x\geq x_0 \end{cases}
\end{align}
\end{subequations}
at some very small negative time $-\tau$ with $0<\tau\ll 1$.
This data contains a jump discontinuity that spreads out as a 1-rarefaction
wave, so that solving up to time $t=0$ gives smooth functions
$\epsilon(x,0)$ and $u(x,0)$.  Now negate $u(x,0)$ and consider the
resulting functions as data at $t=0$.  By time-reversibility, over the time
$0\leq t\leq \tau$ the solution sharpens back into the discontinuity we
started with, corresponding to shock formation.

However, the fact that we can construct data resulting in shock waves 
does not preclude the possibility that smooth solutions
exist for all time for some restricted set of initial data.
%, such as data with sufficiently small derivative everywhere.  

%We have not succeeded in
%proving conditions that guarantee smooth solutions for all time,
%although the computational eveidence suggests that this is possible.

%{\bf I think it may be possible to prove here a negative result that 
%includes continuously varying media, along
%the following lines: given any fixed medium, if we use a steep enough
%initial condition, shocks will form after a short time.}



Detecting the formation of shocks in computed solutions is challenging,
since the computation produces only a finite number of values (cell-averages
in the case of the finite volume methods used here) and in general
it is impossible to determine if the solution is smooth based on these
values.  In practice, one can only expect to obtain an upper-bound on the
magnitude of possible discontinuities.
In \cite{simpson2010}, visual detection of shocks in spectral solutions
was conducted by looking for highly oscillatory regions near steep gradients
(evidence of the Gibbs phenomenon).
%since shocks represent discontinuities, which cannot be represented exactly
%on a finite grid.  Thus, 
%A sufficiently steep gradient is indistinguishable
%from a shock on a finite grid.  By using a very fine computational
%grid, one can potentially rule out the formation of shocks up to some high
%precision.  
Since the solutions we are interested in contain highly oscillatory regions
and steep smooth regions, inspection of the solution may not be a reliable
way to judge whether shocks have formed.  We propose a more robust and
quantitative approach based on two
signatures of shock formation.  The first, mentioned already in the 
introduction, is the {\em loss of time-reversibility}.  The second is 
{\em entropy decay}.  
We now describe the design of experiments that can detect these
signatures.
