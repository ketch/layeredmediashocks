The main findings of this work can be summarized as follows:
\begin{enumerate}
  \item Layered periodic materials with varying impedance tend to inhibit 
        shock formation.
  \item Weak shocks in periodic media are unstable, in the sense that they
        do not persist as noticeable discontinuities and do not lead to significant 
        long-term entropy decay.
  \item Initial perturbations with large enough amplitude do lead to shock
        formation and substantial sustained entropy decay.
  \item The formation of shocks can be quantitatively predicted, to a good
        approximation, by a generalized characteristic condition.
\end{enumerate}

In order to detect the formation of shocks in layered periodic media, 
we have studied the time-reversibility and the entropy
evolution of computated solutions.  Based on these experiments and
some intuition, we are led to a simple criterion for shock formation
in terms of one parameter: the relative effective shock speed $S_\textup{eff}$.  
This theory is consistent with a wide range of experiments involving simple 
layered media.  It also seems to be a natural
generalization of the Lax entropy condition, taking into account the 
effective properties of the periodic medium.

This result has potentially important implications both for nonlinear
hyperbolic PDE theory (the fact that shocks appear to be avoided for all
time when $S_\textup{eff}<1$) and for effective medium theory (the fact
that classical shocks may still form in a periodic medium when $S_\textup{eff}>1$).
The result has been formulated in terms of general periodic media and could
be interpreted in a natural way for other heterogeneous (non-periodic)
media.  Whether the same condition for shock formation holds in such 
broader settings is an open question.
Another way in which this theory could be generalized naturally involves
application to more general first-order hyperbolic systems.
Both of these topics are the subject of ongoing research.

A very interesting question touched on in the introduction is that of 
whether it is possible for non-trivial initial conditions to remain smooth 
for all time in the solution of nonlinear hyperbolic PDEs with varying 
coefficients.
Because the approach of the present work has been based on computation and 
observation, it provides tantalizing hints but does not address 
in a strict way the answer to this question.

We finally remark that the interesting behavior of different limiters
with respect to entropy evolution for the waves studied here seems
worthy of further study.

%Computational results indicate that random media also suppress shock 
%formation \cite{fouque2004time}, although they only delay the formation of shocks
%in a general sense.  The effect seen here is much more remarkable.
%Generalization and application of the shock-formation criterion developed
%in this paper is the subject of future work.

%Further work to understand these waves from a computational perspective
%may include development and application of entropy-conserving schemes,
%as well as very high resolution computations in order to bound the
%entropy production.

\subsection*{Acknowledgments}
This research was supported in part by NSF grant DMS-0914942 and NIH grant
5R01AR53652-2.

